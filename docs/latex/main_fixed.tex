\UseRawInputEncoding

\documentclass[a4paper,12pt]{article}

\usepackage[utf8]{inputenc}

\usepackage[T5]{fontenc}

\usepackage[vietnamese]{babel}

\usepackage{graphicx}

\usepackage{listings}

\usepackage{xcolor}

\usepackage[hidelinks]{hyperref}

\usepackage{bookmark}

\usepackage{float}

\usepackage{setspace}



\DeclareUnicodeCharacter{2228}{$\lor$}

\DeclareUnicodeCharacter{2227}{$\land$}

\DeclareUnicodeCharacter{2265}{$\ge$}

\DeclareUnicodeCharacter{207F}{$^n$}



% --- TOML ---

\lstdefinelanguage{toml}{

  keywords={true,false},

  sensitive=true,

  comment=[l]{\#},

  morestring=[b]"

}



% --- YAML ---

\lstdefinelanguage{yaml}{

  keywords={true,false,null},

  sensitive=true,

  comment=[l]{\#},

  morestring=[b]",

  morestring=[b]'

}



% Cấu hình code nền đen

\lstset{

  backgroundcolor=\color{black},

  basicstyle=\ttfamily\color{white}\footnotesize,

  breaklines=true,

  frame=single,

  rulecolor=\color{gray},

  keywordstyle=\color{cyan},

  stringstyle=\color{yellow},

  showstringspaces=false

}



\begin{document}

\begin{titlepage}

\begin{center}



\onehalfspacing

{\Large \textbf{ĐẠI HỌC QUỐC GIA THÀNH PHỐ HỒ CHÍ MINH}}\\[0.25cm]

{\large \textbf{TRƯỜNG ĐẠI HỌC CÔNG NGHỆ THÔNG TIN}}\\[0.25cm]

{\large \textbf{KHOA MẠNG MÁY TÍNH VÀ TRUYỀN THÔNG}}\\[1cm]



%\includegraphics[width=0.25\linewidth]{uit_logo.jpg}\\[1.0cm] 

{\Large \textbf{BÁO CÁO TIẾN ĐỘ}}\\[0.5cm]

{\large \textbf{MÔN HỌC: MẬT MÃ HỌC}}\\[1.0cm]

{\large \textbf{Đề tài: Ứng dụng Zero Knowledge Proofs và thuật

toán hậu lượng tử (Post Quantum Cryptography) vào việc bảo vệ tài sản và xác minh danh tính trong mạng Blockchain liên

ngân hàng}}\\[0.5cm]

\begin{tabular}{rl}

\textbf{Giảng viên hướng dẫn:} & TS. Nguyễn Ngọc Tự\\[0.3cm]

\textbf{Lớp:} & NT219.Q12.ANTT\\[0.3cm]

\textbf{Sinh viên thực hiện:} & Nguyễn Hoàng Quý – 24521494\\

& Huỳnh Nhật Duy – 24520375\\

\end{tabular}\\[2.5cm]



\textit{Thành phố Hồ Chí Minh, tháng 09 năm 2025}



\end{center}

\end{titlepage}



\newpage

\tableofcontents

\newpage

\section{MỞ ĐẦU}

\par

Công nghệ \textbf{Blockchain} đã trở thành nền tảng quan trọng trong lĩnh vực tài chính, đặc biệt là đối với các hệ thống ngân hàng và liên ngân hàng, nhờ vào các đặc tính nổi bật như \textit{tính minh bạch}, \textit{toàn vẹn dữ liệu} và \textit{phi tập trung}. Tuy nhiên, sự phát triển nhanh chóng của \textbf{máy tính lượng tử} \cite{mavroeidis2018} đang đặt ra thách thức lớn đối với các thuật toán mật mã hiện hành như \textbf{RSA} và \textbf{ECDSA}. Những thuật toán này có thể bị phá vỡ khi năng lực tính toán lượng tử đạt đến ngưỡng nhất định, dẫn đến nhu cầu cấp thiết trong việc nghiên cứu và triển khai các \textbf{thuật toán mật mã hậu lượng tử (PQC)} để đảm bảo an toàn cho các giao dịch trên blockchain.



\par

Bên cạnh mối đe dọa từ máy tính lượng tử, các quy trình nghiệp vụ truyền thống trong ngân hàng, chẳng hạn như \textit{rút tiền} hay \textit{sao kê tài khoản}, vẫn còn phức tạp và tiềm ẩn nhiều rủi ro bảo mật. Cụ thể, khách hàng thường phải cung cấp các giấy tờ định danh như \textbf{CCCD} hoặc \textbf{số dư} trong mỗi giao dịch, làm tăng nguy cơ \textbf{rò rỉ thông tin cá nhân}. Điều này không chỉ ảnh hưởng đến quyền riêng tư của người dùng mà còn có thể dẫn đến việc \textbf{kẻ tấn công lợi dụng dữ liệu bị rò rỉ để tạo và ghi nhận các giao dịch giả mạo} trong hệ thống.

\section{PHÂN TÍCH HỆ THỐNG}

\subsection{Ngữ cảnh}



Các ngân hàng thương mại cung cấp nhiều dịch vụ tài chính, trong đó có quản lý tài khoản và xử lý các giao dịch của khách hàng. Khi phát sinh yêu cầu chuyển tiền liên ngân hàng hoặc thanh toán, hệ thống của ngân hàng nhận (hoặc đơn vị trung gian) phải kiểm tra tính hợp lệ của giao dịch và xác minh danh tính người gửi.



Việc chứng thực nguồn gốc cũng như đảm bảo tính toàn vẹn của gói tin là điều bắt buộc nhằm tránh thất thoát tài sản hoặc chuyển nhầm đối tượng. Tuy nhiên, các cơ chế ký số phổ biến hiện nay như RSA hay ECDSA đang đối mặt với rủi ro bị bẻ khóa bởi máy tính lượng tử trong tương lai gần. Điều này đặt ra nhu cầu về một giải pháp xác thực mới, có khả năng chống lại năng lực tính toán lượng tử, đồng thời không làm lộ thông tin định danh cá nhân (PII) của khách hàng trong suốt quá trình kiểm tra.

\begin{figure}

    \centering

    \includegraphics[width=1\linewidth]{context.png}

    \caption{Ngân hàng hiện tại}

    \label{fig:placeholder}

\end{figure}

\subsection{Các bên liên quan}



\begin{itemize}

    \item \textbf{Ngân hàng gửi}: Khởi tạo, ký và truyền gói giao dịch; đảm bảo thông tin chính xác và không lộ dữ liệu khách hàng.



    \item \textbf{Ngân hàng nhận}: Tiếp nhận và xác minh tính hợp lệ của giao dịch; kiểm tra chữ ký, phát hiện giả mạo và đảm bảo an toàn.



    \item \textbf{Khách hàng}: Chủ tài khoản gửi yêu cầu giao dịch; cần được bảo vệ thông tin cá nhân và tính toàn vẹn giao dịch.



    \item \textbf{Hạ tầng mật mã}: Quản lý khóa và thuật toán hậu lượng tử; duy trì tính an toàn và toàn vẹn trong quy trình xác thực.

\end{itemize}



\section{Tài sản cần bảo vệ }



\subsection{Dữ liệu khách hàng}

Bảo vệ PII và thông tin tài chính như CCCD, email, số dư, lịch sử giao dịch; tránh truy cập trái phép khi lưu trữ và truyền tải.



\subsection{Tính toàn vẹn và xác thực giao dịch}

Đảm bảo giao dịch xuất phát từ đúng chủ tài khoản và dữ liệu trên sổ cái không bị thay đổi sau khi ghi.



\subsection{Thông tin bảo mật về khóa}

Bảo vệ khóa riêng của ngân hàng khỏi rò rỉ, tấn công vét cạn và đảm bảo khả năng kháng lượng tử.



\section{Mục tiêu bảo mật (Security Objectives)}



\subsection{Xác minh giao dịch}

Đảm bảo giao dịch hợp lệ thông qua xác thực chữ ký PQC Dilithium, kiểm tra số dư và ngăn chi tiêu lặp. Nonce và chữ ký giúp chống phát lại giữa các mạng.



\subsection{Tính bất biến của sổ cái}

Dữ liệu sau khi ghi vào khối không thể bị sửa đổi nhờ liên kết hash. ZSTARKs cho phép xác minh mà không tiết lộ thông tin nhạy cảm.



\subsection{Kháng tấn công lượng tử}

Sử dụng thuật toán hậu lượng tử như Dilithium để tạo và quản lý khóa, đảm bảo giao dịch vẫn an toàn trước máy tính lượng tử.



\subsection{Tăng cường quyền riêng tư}

ZKP giúp chứng minh quyền sở hữu hoặc điều kiện KYC mà không lộ PII. ZSTARKs che giấu thông tin giao dịch nhưng vẫn đảm bảo tính hợp lệ.



\section{GIẢI PHÁP}







\section{TRIỂN KHAI HỆ THỐNG}





\subsubsection{Yêu cầu phần cứng}



\begin{itemize}

    \item CPU: Intel Core i7 hoặc tương đương ($\geq$ 4 cores, khuyến nghị 8 cores)

    \item RAM: Tối thiểu 16GB (khuyến nghị 32GB)

    \item Ổ cứng: SSD tối thiểu 100GB trống

    \item Network: Băng thông $\geq$ 100 Mbps

\end{itemize}



\paragraph{Lý do lựa chọn.}

Cấu hình nhóm em lựa chọn để phản ánh tương đối sát môi trường vận hành của một hệ thống blockchain liên ngân hàng thực tế, nơi các node validator, KSM và prover đều tiêu tốn tài nguyên CPU và RAM đáng kể do phải xử lý chữ ký PQC và sinh/kiểm tra bằng chứng ZK. SSD và băng thông mạng cao giúp đảm bảo độ trễ thấp và thông lượng ổn định khi kiểm thử các kịch bản tải giao dịch và ZK-Rollup như đã nêu trong \cite{NT219_BaoCaoTienDo2}.



% ----------------------------

\subsubsection{Hệ điều hành}



\begin{lstlisting}[language=bash]

Ubuntu 24.04 LTS

lsb_release -a

\end{lstlisting}



\paragraph{Lý do lựa chọn.}

Ubuntu 24.04 LTS được chọn vì đây là bản phân phối Linux ổn định, phổ biến trong môi trường server và dễ tích hợp với Docker, Besu, Java, Node.js cũng như các thư viện mật mã hậu lượng tử. Việc dùng cùng một hệ điều hành cho tất cả node trong mạng consortium giúp giảm sai lệch cấu hình, thống nhất môi trường triển khai và bám sát giả định về hạ tầng trong báo cáo tiến độ.



% ----------------------------

\subsubsection{Cài đặt Docker và Docker Compose}



\begin{lstlisting}[language=bash]

sudo apt update

sudo apt install -y apt-transport-https ca-certificates curl software-properties-common

curl -fsSL https://download.docker.com/linux/ubuntu/gpg | sudo gpg --dearmor -o /usr/share/keyrings/docker.gpg

sudo apt install -y docker-ce docker-compose-plugin

docker --version

docker compose version

\end{lstlisting}



\paragraph{Lý do lựa chọn.}

Docker và Docker Compose cho phép nhóm đóng gói từng thành phần (Besu node, KSM, các dịch vụ phụ trợ) thành container, từ đó dễ dàng khởi tạo, dừng và tái lập toàn bộ môi trường thí nghiệm. Đây là lựa chọn phù hợp với kiến trúc consortium blockchain được mô tả trong báo cáo, nơi nhiều node và dịch vụ phải phối hợp với nhau nhưng vẫn cần tách biệt và quản lý độc lập.



% ----------------------------

\subsubsection{Cài đặt Node.js}



\begin{lstlisting}[language=bash]

curl -fsSL https://deb.nodesource.com/setup_18.x | sudo -E bash -

sudo apt install -y nodejs

node --version

npm --version

\end{lstlisting}



\paragraph{Lý do lựa chọn.}

Node.js được sử dụng để triển khai các script biên dịch \emph{Smart Contract}, deploy contract lên Hyperledger Besu và dựng Web GUI. Hệ sinh thái npm cung cấp sẵn các thư viện tương tác EVM (như ethers.js), giúp rút gọn đáng kể công sức tích hợp giữa blockchain core và lớp ứng dụng, phù hợp với mục tiêu xây dựng nhanh một prototype đầy đủ luồng giao dịch như trong \cite{NT219_BaoCaoTienDo2}.



% ----------------------------

\subsubsection{Cài đặt Java OpenJDK}



\begin{lstlisting}[language=bash]

sudo apt install -y openjdk-17-jdk

java --version

\end{lstlisting}



\paragraph{Lý do lựa chọn.}

Java OpenJDK 17 được chọn làm nền tảng cho Key Simulation Module (KSM) vì Java quen thuộc trong môi trường doanh nghiệp/ngân hàng, có hệ sinh thái phong phú cho lập trình mạng, bảo mật và tích hợp với các thư viện PQC. Điều này phù hợp với định hướng của đề tài là mô phỏng một HSM phần mềm an toàn, có khả năng quản lý khóa và ký giao dịch bằng thuật toán hậu lượng tử.



% ----------------------------

\subsubsection{Cài đặt Maven}



\begin{lstlisting}[language=bash]

sudo apt install -y maven

mvn --version

\end{lstlisting}



\paragraph{Lý do lựa chọn.}

Maven được sử dụng để quản lý vòng đời build của KSM, tự động hóa việc tải dependencies, biên dịch và đóng gói ứng dụng Java. Việc dùng Maven giúp chuẩn hóa quy trình build, dễ tích hợp vào pipeline CI/CD trong tương lai và phù hợp với cách tổ chức dự án Java trong các hệ thống nghiệp vụ ngân hàng.



% ----------------------------

\subsubsection{Kiểm tra OpenSSL}



\begin{lstlisting}[language=bash]

openssl version

\end{lstlisting}



\paragraph{Lý do lựa chọn.}

OpenSSL là công cụ chuẩn để kiểm tra thư viện mật mã và hỗ trợ cấu hình, debug TLS 1.3. Trong bối cảnh đề tài, OpenSSL được dùng để xác nhận phiên bản, kiểm thử bắt tay TLS và đảm bảo rằng môi trường triển khai đáp ứng được yêu cầu bảo mật kênh truyền (mTLS, TLS 1.3) đã phân tích trong phần mục tiêu bảo mật.



% ----------------------------

\subsection{Triển khai TLS 1.3}



\subsubsection{Tạo Certificate Authority}



\begin{lstlisting}[language=bash]

./scripts/generate_tls13_certs.sh

\end{lstlisting}



\paragraph{Lý do lựa chọn.}

Nhóm tự xây dựng một Certificate Authority nội bộ và script sinh chứng chỉ TLS 1.3 để mô phỏng bối cảnh consortium, nơi các ngân hàng thành viên chia sẻ cùng một hệ thống tin cậy. Cách tiếp cận này cho phép cấu hình \emph{mutual TLS} giữa các node mà không phụ thuộc vào CA bên ngoài, đồng thời vẫn đảm bảo kiểm soát toàn bộ chuỗi chứng chỉ trong môi trường thí nghiệm.



\subsubsection{Cấu hình TLS cho node}



\begin{lstlisting}[language=toml]

rpc-http-tls-enabled=true

rpc-http-tls-protocols=["TLSv1.3"]

rpc-http-tls-cipher-suites=["TLS_AES_256_GCM_SHA384"]

\end{lstlisting}



\paragraph{Lý do lựa chọn.}

Besu được cấu hình bắt buộc dùng TLS 1.3 với bộ mã \texttt{TLS\_AES\_256\_GCM\_SHA384} nhằm đảm bảo mức bảo mật hiện đại và phù hợp với thông lệ trong môi trường tài chính. Việc ép buộc kênh RPC chạy trên TLS 1.3 giúp bảo vệ payload giao dịch và proof ZK, bám sát yêu cầu "in-transit security" và kháng tấn công nghe lén, chỉnh sửa gói tin như báo cáo đã phân tích.



% ----------------------------

\subsection{Triển khai Key Simulation Module (KSM)}



\subsubsection{Build KSM}



\begin{lstlisting}[language=bash]

mvn clean package -DskipTests

\end{lstlisting}



\paragraph{Lý do lựa chọn.}

Quy trình build bằng Maven giúp đóng gói KSM thành một artefact rõ ràng, dễ triển khai trong container hoặc trên máy chủ. Việc tách bước build KSM ra riêng phản ánh đúng kiến trúc đề xuất, nơi KSM đóng vai trò một thành phần độc lập chuyên trách quản lý và ký bằng khóa PQC.



\subsubsection{Cấu hình lưu trữ khóa}



\begin{lstlisting}[language=bash]

mkdir -p ksm-data/keys

openssl rand -hex 32 > ksm-data/master.key

chmod 600 ksm-data/master.key

\end{lstlisting}



\paragraph{Lý do lựa chọn.}

Việc tách riêng thư mục lưu trữ khóa và sử dụng \texttt{master.key} với quyền truy cập hạn chế mô phỏng cách HSM hoặc module quản lý khóa trong thực tế bảo vệ bí mật khóa. Cách cấu hình này phù hợp với mục tiêu bảo vệ "thông tin bảo mật về khóa" trong báo cáo, giảm nguy cơ rò rỉ trực tiếp khóa riêng của ngân hàng.



\subsubsection{Docker Compose cho KSM}



\begin{lstlisting}[language=yaml]

ksm:

  build: ../ksm

  ports:

    - "8080:8080"

  volumes:

    - ./ksm-data:/app/ksm-data

\end{lstlisting}



\paragraph{Lý do lựa chọn.}

Đóng gói KSM thành một service Docker riêng cho phép nhóm triển khai, nâng cấp hoặc thay thế KSM mà không ảnh hưởng tới các node blockchain. Điều này phản ánh tinh thần \emph{crypto-agility} trong kiến trúc đề xuất, nơi lớp quản lý khóa PQC có thể được xoay vòng hoặc nâng cấp thuật toán mà vẫn giữ nguyên phần còn lại của hệ thống.



\subsubsection{Khởi động KSM}



\begin{lstlisting}[language=bash]

docker compose up -d ksm

docker compose logs -f ksm

\end{lstlisting}



\paragraph{Lý do lựa chọn.}

Việc khởi động KSM bằng Docker Compose và theo dõi log giúp dễ dàng giám sát trạng thái module khóa, đảm bảo rằng KSM sẵn sàng phục vụ các yêu cầu ký/xác minh trước khi ghép nối với Besu và Web GUI. Điều này giảm lỗi trong quá trình kiểm thử luồng giao dịch PQC thực tế.



\subsubsection{Test KSM API}



\begin{lstlisting}[language=bash]

curl http://localhost:8080/ksm/health

\end{lstlisting}



\paragraph{Lý do lựa chọn.}

Endpoint \texttt{/ksm/health} cho phép kiểm tra nhanh tình trạng KSM theo kiểu \emph{health check}, giúp xác định sớm sự cố trước khi đưa vào chuỗi giao dịch liên ngân hàng. Đây cũng là thực hành phổ biến khi triển khai microservice trong môi trường sản xuất.



% ----------------------------

\subsection{Triển khai Hyperledger Besu}



\subsubsection{Khởi động mạng blockchain}



\begin{lstlisting}[language=bash]

docker compose up -d

docker compose ps

\end{lstlisting}



\paragraph{Lý do lựa chọn.}

Hyperledger Besu được chọn vì hỗ trợ EVM, cơ chế đồng thuận IBFT2/QBFT và tích hợp tốt với kiến trúc consortium blockchain mà đề tài hướng tới. Khởi động mạng bằng Docker Compose cho phép dựng nhanh nhiều node validator, dễ dàng mô phỏng kịch bản nhiều ngân hàng thành viên cùng tham gia xác thực và đồng thuận trên ledger chung.



\subsubsection{Kiểm tra TLS 1.3}



\begin{lstlisting}[language=bash]

openssl s_client -connect localhost:21001 -tls1_3

\end{lstlisting}



\paragraph{Lý do lựa chọn.}

Lệnh \texttt{openssl s\_client} giúp kiểm tra trực tiếp việc bắt tay TLS 1.3 với node Besu, xác nhận rằng cấu hình chứng chỉ và phiên bản giao thức là chính xác. Đây là bước quan trọng để đảm bảo rằng tất cả giao tiếp RPC với Besu đều được mã hóa, phù hợp với các mục tiêu bảo mật kênh truyền đã nêu.



% ----------------------------

\subsection{Triển khai Smart Contract}



\subsubsection{Compile contract}



\begin{lstlisting}[language=bash]

npm install --legacy-peer-deps

node scripts/compile.js

\end{lstlisting}



\paragraph{Lý do lựa chọn.}

Việc biên dịch Smart Contract bằng script Node.js cho phép nhóm kiểm soát chặt chẽ phiên bản compiler, đường dẫn output ABI/bytecode và tích hợp dễ dàng với các bước deploy sau đó. Đây là bước chuẩn bị cần thiết để xây dựng \emph{Verifier Contract} và các contract nghiệp vụ khác trên nền Besu.



\subsubsection{Deploy contract}



\begin{lstlisting}[language=bash]

node scripts/public/deploy_and_init.js

\end{lstlisting}



\paragraph{Lý do lựa chọn.}

Script \texttt{deploy\_and\_init.js} tự động hoá cả quá trình deploy và khởi tạo contract (thiết lập số dư, phân quyền, v.v.), giúp đảm bảo mỗi lần reset blockchain đều tái tạo được một trạng thái nhất quán để kiểm thử. Điều này phù hợp với yêu cầu lặp lại các thí nghiệm trong báo cáo tiến độ và giảm lỗi thao tác thủ công.



% ----------------------------

\subsection{Tích hợp PQC với Transaction Flow}



\subsubsection{Sinh khóa PQC}



\begin{lstlisting}[language=bash]

curl -X POST http://localhost:8080/ksm/generateKeyPair

\end{lstlisting}



\paragraph{Lý do lựa chọn.}

Việc sinh cặp khóa PQC thông qua API của KSM tách biệt hoàn toàn việc quản lý khóa khỏi ứng dụng và node blockchain, mô phỏng mô hình HSM thực tế nơi ví hoặc ngân hàng không trực tiếp giữ khóa riêng. Cách tiếp cận này hỗ trợ tốt mục tiêu bảo vệ PII và tài sản số của khách hàng khỏi rủi ro rò rỉ khóa.



\subsubsection{Ký và xác minh}



\begin{lstlisting}[language=bash]

curl -X POST http://localhost:8080/ksm/sign

curl -X POST http://localhost:8080/ksm/verify

\end{lstlisting}



\paragraph{Lý do lựa chọn.}

Các endpoint \texttt{/sign} và \texttt{/verify} cho phép kiểm thử luồng ký/xác minh PQC độc lập với blockchain, từ đó đánh giá được độ trễ, kích thước chữ ký và hành vi an toàn trước khi tích hợp sâu vào Besu. Điều này bám sát kế hoạch đánh giá hiệu năng và bảo mật (E-Crypto, E-AuthN, E-AuthZ) được đề cập trong tài liệu tham khảo.



% ----------------------------

\subsection{Triển khai Web GUI}



\subsubsection{Khởi động giao diện}



\begin{lstlisting}[language=bash]

npm run dev

\end{lstlisting}



\paragraph{Lý do lựa chọn.}

Web GUI được xây dựng bằng stack JavaScript hiện đại giúp người dùng (ngân hàng, khách hàng) trực quan hoá toàn bộ luồng giao dịch PQC và ZKP trên nền blockchain liên ngân hàng. Việc tách giao diện ra khỏi backend Besu/KSM làm rõ kiến trúc phân lớp, đồng thời thuận tiện cho việc trình bày và demo kết quả trong khuôn khổ môn học.



% ----------------------------

\subsection{Kiểm thử hệ thống}



\subsubsection{Hiệu năng}



\begin{itemize}

    \item Thời gian ký Dilithium3: $\sim$5ms

    \item Kích thước chữ ký: $\sim$2420 bytes

    \item TPS: 30--40 giao dịch/giây

\end{itemize}



\paragraph{Lý do lựa chọn.}

Các chỉ số thời gian ký, kích thước chữ ký và TPS phản ánh trực tiếp chi phí thực tế của việc tích hợp PQC vào luồng giao dịch blockchain. Đây là các thước đo quan trọng để đánh giá liệu hệ thống đề xuất có đáp ứng được yêu cầu vận hành trong bối cảnh ngân hàng hay không, nhất là khi kích thước chữ ký PQC lớn hơn đáng kể so với chữ ký ECDSA truyền thống.



\subsubsection{Bảo mật}



\begin{itemize}

    \item Reject chữ ký sai: 100\%

    \item Reject sai khóa: $\geq$99\%

    \item Chống replay attack bằng nonce

\end{itemize}



\paragraph{Lý do lựa chọn.}

Các tiêu chí kiểm thử bảo mật tập trung vào việc đảm bảo mọi bằng chứng hoặc chữ ký không hợp lệ đều bị từ chối, đồng thời ngăn chặn tấn công phát lại thông qua cơ chế nonce. Đây là những yêu cầu cốt lõi để đảm bảo tính toàn vẹn, xác thực và không chối bỏ của giao dịch, bám sát mục tiêu bảo mật đã đề ra trong báo cáo tiến độ.



% ----------------------------

% ============================================

% PHẦN BỔ SUNG: Triển khai PKI Registry, ZKP Prover, và PQCSignatureRegistry

% ============================================



\subsection{Triển khai PKI Registry}



\subsubsection{Deploy PKI Registry Contract}



\begin{lstlisting}[language=bash]

cd Besu-hyperledger/smart_contracts

node scripts/compile.js

export NODE_TLS_REJECT_UNAUTHORIZED=0

RPC_ENDPOINT=https://localhost:21001 node scripts/deploy_pki.js

unset NODE_TLS_REJECT_UNAUTHORIZED

\end{lstlisting}



\paragraph{Lý do lựa chọn.}



PKI Registry được triển khai như một contract riêng biệt để quản lý danh tính người dùng, xác minh KYC và kiểm soát quyền truy cập. Việc tách biệt PKI khỏi contract chính giúp giảm kích thước bytecode của \texttt{InterbankTransfer}, đồng thời cho phép nâng cấp hoặc thay thế module PKI mà không ảnh hưởng đến logic nghiệp vụ chính. Cách tiếp cận này phù hợp với nguyên tắc \emph{separation of concerns} và hỗ trợ tốt mục tiêu bảo vệ PII thông qua việc chỉ lưu hash của dữ liệu KYC thay vì dữ liệu thực tế.



\subsubsection{Fund Users và Register vào PKI}



\begin{lstlisting}[language=bash]

# Fund users with native ETH for gas fees

RPC_ENDPOINT=https://localhost:21001 node scripts/fund_users_for_pki.js



# Register all users to PKI Registry

RPC_ENDPOINT=https://localhost:21001 node scripts/register_all_users_pki.js

\end{lstlisting}



\paragraph{Lý do lựa chọn.}



Quy trình hai bước (fund + register) phản ánh đúng luồng hoạt động trong môi trường thực tế, nơi người dùng cần có native ETH để trả phí gas khi tự đăng ký vào hệ thống PKI. Script \texttt{register\_all\_users\_pki.js} tự động hoá việc đăng ký nhiều users cùng lúc, giúp giảm thời gian setup và đảm bảo tính nhất quán trong quá trình kiểm thử.



\subsubsection{Link PKI vào InterbankTransfer}



\begin{lstlisting}[language=bash]

RPC_ENDPOINT=https://localhost:21001 node scripts/link_pki_interbank.js

\end{lstlisting}



\paragraph{Lý do lựa chọn.}



Việc link PKI Registry vào \texttt{InterbankTransfer} thông qua một script riêng cho phép kiểm soát chặt chẽ thứ tự triển khai và đảm bảo rằng cả hai contracts đều đã được deploy thành công trước khi thiết lập mối quan hệ giữa chúng. Cách tiếp cận này giúp dễ dàng debug và rollback nếu có lỗi xảy ra trong quá trình tích hợp.



% ----------------------------



\subsection{Triển khai ZKP Prover Service}



\subsubsection{Cài đặt Rust và Dependencies}



\begin{lstlisting}[language=bash]

curl --proto '=https' --tlsv1.2 -sSf https://sh.rustup.rs | sh

source $HOME/.cargo/env

rustc --version

\end{lstlisting}



\paragraph{Lý do lựa chọn.}



Rust được chọn làm ngôn ngữ triển khai ZKP Prover vì hiệu năng cao và khả năng tích hợp tốt với thư viện \textbf{Winterfell STARK}, một trong những thư viện ZKP phổ biến và được đánh giá cao trong cộng đồng. Rust cũng đảm bảo an toàn bộ nhớ mà không cần garbage collector, phù hợp với các ứng dụng yêu cầu độ trễ thấp như việc sinh proof ZK.



\subsubsection{Build ZKP Prover Service}



\begin{lstlisting}[language=bash]

cd prover

cargo build --release

\end{lstlisting}



\paragraph{Lý do lựa chọn.}



Build ở chế độ \texttt{--release} với các tối ưu hóa (LTO, codegen-units=1) giúp giảm kích thước binary và tăng tốc độ thực thi, điều quan trọng khi prover service cần xử lý nhiều yêu cầu sinh proof đồng thời. Việc tách biệt prover thành một service độc lập cho phép scale theo chiều ngang và không làm chậm các node blockchain.



\subsubsection{Khởi động ZKP Prover}



\begin{lstlisting}[language=bash]

RUST_LOG=info ./target/release/zkp-prover \&

\end{lstlisting}



\paragraph{Lý do lựa chọn.}



ZKP Prover chạy như một service độc lập trên port 8081, cung cấp REST API để các ứng dụng khác (GUI, smart contract clients) gọi để sinh proof. Việc chạy ở background (\texttt{\&}) cho phép prover hoạt động song song với các thành phần khác của hệ thống, phù hợp với kiến trúc microservice và đảm bảo rằng việc sinh proof không block các luồng xử lý khác.



\subsubsection{Test ZKP Prover API}



\begin{lstlisting}[language=bash]

curl http://localhost:8081/health | python3 -m json.tool

\end{lstlisting}



\paragraph{Lý do lựa chọn.}



Endpoint \texttt{/health} cho phép kiểm tra nhanh tình trạng của ZKP Prover service, giúp xác định sớm các sự cố trước khi tích hợp vào luồng giao dịch. Đây là thực hành phổ biến trong kiến trúc microservice và giúp đảm bảo tính sẵn sàng của hệ thống.



% ----------------------------



\subsection{Triển khai Balance Verifier Contract}



\subsubsection{Deploy BalanceVerifier}



\begin{lstlisting}[language=bash]

cd Besu-hyperledger/smart_contracts

export NODE_TLS_REJECT_UNAUTHORIZED=0

RPC_ENDPOINT=https://localhost:21001 node scripts/deploy_balance_verifier.js

unset NODE_TLS_REJECT_UNAUTHORIZED

\end{lstlisting}



\paragraph{Lý do lựa chọn.}



\texttt{BalanceVerifier} được triển khai như một contract riêng để thực hiện việc xác minh proof ZKP on-chain. Contract này chỉ thực hiện các kiểm tra integrity (hash matching, public inputs validation) thay vì full STARK verification để giảm chi phí gas và đảm bảo hiệu năng cao. Việc tách biệt verifier khỏi contract chính giúp giảm kích thước bytecode của \texttt{InterbankTransfer} và tuân thủ giới hạn EIP-170.



\subsubsection{Link BalanceVerifier vào InterbankTransfer}



\begin{lstlisting}[language=bash]

RPC_ENDPOINT=https://localhost:21001 node scripts/set_balance_verifier.js

RPC_ENDPOINT=https://localhost:21001 node scripts/toggle_zkp.js

\end{lstlisting}



\paragraph{Lý do lựa chọn.}



Việc link \texttt{BalanceVerifier} vào \texttt{InterbankTransfer} và bật flag \texttt{zkpEnabled} cho phép contract chính sử dụng ZKP để xác minh điều kiện \texttt{balance > amount} mà không tiết lộ giá trị balance thực tế. Cách tiếp cận này đảm bảo tính riêng tư của người dùng trong khi vẫn duy trì tính toàn vẹn và bảo mật của giao dịch, phù hợp với mục tiêu "Tăng cường quyền riêng tư" đã đề ra trong báo cáo tiến độ.



% ----------------------------



\subsection{Triển khai PQCSignatureRegistry Contract}



\subsubsection{Deploy PQCSignatureRegistry}



\begin{lstlisting}[language=bash]

cd Besu-hyperledger/smart_contracts

node scripts/compile.js

# Deploy PQCSignatureRegistry (if separate script exists)

# Or deploy manually via Remix/Hardhat

\end{lstlisting}



\paragraph{Lý do lựa chọn.}



\texttt{PQCSignatureRegistry} được triển khai như một contract riêng để lưu trữ chữ ký PQC on-chain cho từng transaction, cho phép audit và verify sau này. Việc tách biệt registry khỏi contract chính giúp giảm kích thước bytecode của \texttt{InterbankTransfer} và tuân thủ giới hạn EIP-170, đồng thời vẫn đảm bảo yêu cầu lưu PQC signature/hash on-chain như đã phân tích trong báo cáo tiến độ.



\subsubsection{Link PQCSignatureRegistry vào InterbankTransfer}



\begin{lstlisting}[language=bash]

# Call setPQCRegistry(...) on InterbankTransfer to link registry

# (execute via script or Remix)

\end{lstlisting}



\paragraph{Lý do lựa chọn.}



Việc link \texttt{PQCSignatureRegistry} vào \texttt{InterbankTransfer} cho phép contract chính gọi \texttt{storePQCSignature} để lưu chữ ký PQC on-chain mỗi khi có giao dịch với PQC. Cách tiếp cận này đảm bảo tính toàn vẹn và khả năng audit của hệ thống trong khi vẫn giữ kích thước contract chính ở mức hợp lý.



% ----------------------------



\subsection{Tích hợp ZKP vào Transaction Flow}



\subsubsection{Cấu hình GUI để sử dụng ZKP}



\begin{lstlisting}[language=bash]

cd GUI/web

echo "NEXT_PUBLIC_ZKP_PROVER_URL=http://localhost:8081" >> .env.local

\end{lstlisting}



\paragraph{Lý do lựa chọn.}



Việc cấu hình URL của ZKP Prover thông qua biến môi trường cho phép GUI biết địa chỉ của prover service và gọi API để sinh proof trước khi gửi transaction lên blockchain. Cách tiếp cận này đảm bảo tính linh hoạt và dễ dàng thay đổi cấu hình khi triển khai trong các môi trường khác nhau (development, staging, production).



\subsubsection{Luồng giao dịch với ZKP}



\begin{enumerate}

    \item Client (GUI) gọi ZKP Prover API để sinh proof (OFF-CHAIN)

    \item Client nhận proof và gửi transaction với proof hash lên blockchain

    \item Blockchain verify proof hash (ON-CHAIN, nhanh)

    \item Transaction được commit vào block

\end{enumerate}



\paragraph{Lý do lựa chọn.}



Luồng này đảm bảo rằng việc sinh proof ZKP diễn ra hoàn toàn OFF-CHAIN, không làm chậm các node blockchain hoặc block commit. Chỉ proof hash và public inputs được gửi lên blockchain để verify, giảm chi phí gas và đảm bảo hiệu năng cao. Cách tiếp cận này phù hợp với mục tiêu "Scalability" và "Performance" đã đề ra trong báo cáo tiến độ.



% ----------------------------



\subsection{Kiểm thử hệ thống}



\subsubsection{Hiệu năng}



\begin{itemize}

    \item Thời gian ký Dilithium3: $\sim$5ms

    \item Kích thước chữ ký: $\sim$2420 bytes

    \item TPS: 30--40 giao dịch/giây

    \item Thời gian sinh ZKP proof: $\sim$2--5 giây (OFF-CHAIN, không block blockchain)

\end{itemize}



\paragraph{Lý do lựa chọn.}



Các chỉ số hiệu năng phản ánh trực tiếp chi phí thực tế của việc tích hợp PQC và ZKP vào luồng giao dịch blockchain. Mặc dù thời gian sinh ZKP proof khá lâu (2--5 giây), nhưng vì quá trình này diễn ra OFF-CHAIN nên không ảnh hưởng đến hiệu năng của blockchain. Đây là các thước đo quan trọng để đánh giá liệu hệ thống đề xuất có đáp ứng được yêu cầu vận hành trong bối cảnh ngân hàng hay không.



\subsubsection{Bảo mật}



\begin{itemize}

    \item Reject chữ ký sai: 100\%

    \item Reject sai khóa: $\geq$99\%

    \item Chống replay attack bằng nonce

    \item ZKP proof verification: 100\% (chỉ chấp nhận proof hợp lệ)

\end{itemize}



\paragraph{Lý do lựa chọn.}



Các tiêu chí kiểm thử bảo mật tập trung vào việc đảm bảo mọi bằng chứng hoặc chữ ký không hợp lệ đều bị từ chối, đồng thời ngăn chặn tấn công phát lại thông qua cơ chế nonce. Việc thêm ZKP proof verification vào danh sách kiểm thử đảm bảo rằng hệ thống chỉ chấp nhận các proof hợp lệ, bám sát mục tiêu bảo mật đã đề ra trong báo cáo tiến độ.



% ----------------------------



% ----------------------------

\subsection{Performance và Stress Test}

\subsubsection{Công cụ Benchmark}

Nhóm sử dụng \textbf{Lacchain Ethereum-Benchmark}, một công cụ mã nguồn mở chuyên dụng để đánh giá hiệu năng của các blockchain tương thích Ethereum. Công cụ này cho phép kiểm thử tải với nhiều cấu hình khác nhau, đo lường các chỉ số quan trọng như TPS (Transactions Per Second), latency, và success rate.

\paragraph{Lý do lựa chọn.}

Lacchain Ethereum-Benchmark được chọn vì hỗ trợ tốt các smart contract tùy chỉnh, có khả năng tạo nhiều tài khoản và gửi giao dịch song song, phù hợp với kịch bản kiểm thử hệ thống liên ngân hàng với nhiều người dùng đồng thời. Công cụ này cũng tự động quản lý nonce và xử lý các lỗi giao dịch, giúp đảm bảo tính chính xác của kết quả benchmark.

\subsubsection{Thiết lập Benchmark}

\begin{lstlisting}[language=bash]
cd ethereum-benchmark
./prepare-benchmark.sh 200 10
./RUN_BENCHMARK.sh <TPS> <MINUTES>
\end{lstlisting}

\paragraph{Lý do lựa chọn.}

Script \texttt{prepare-benchmark.sh} tự động hóa việc chuẩn bị môi trường benchmark, bao gồm việc tạo và nạp tiền cho 200 tài khoản test, đảm bảo mỗi tài khoản có đủ số dư để thực hiện giao dịch. Việc tách riêng bước chuẩn bị giúp dễ dàng tái sử dụng các tài khoản đã được fund và giảm thời gian setup cho các lần benchmark tiếp theo.

\subsubsection{Kết quả Benchmark}

\paragraph{Test 1 TPS (1 phút).}

\begin{itemize}
    \item \textbf{Success Rate:} 100\%
    \item \textbf{TPS thực tế:} 1.00 tx/s
    \item \textbf{Latency trung bình:} $\sim$4.0 giây
    \item \textbf{Transactions gửi:} 60 tx
    \item \textbf{Transactions thành công:} 60 tx
\end{itemize}

\paragraph{Test 10 TPS (2 phút).}

\begin{itemize}
    \item \textbf{Success Rate:} 100\%
    \item \textbf{TPS thực tế:} 10.01 tx/s
    \item \textbf{Latency trung bình:} $\sim$4.0 giây
    \item \textbf{Transactions gửi:} 1200 tx
    \item \textbf{Transactions thành công:} 1200 tx
\end{itemize}

\paragraph{Test 20 TPS (2 phút).}

\begin{itemize}
    \item \textbf{Success Rate:} 0.00\%
    \item \textbf{TPS thực tế:} 0.00 tx/s
    \item \textbf{Transactions gửi:} 2400 tx
    \item \textbf{Transactions thành công:} 0 tx
    \item \textbf{Nguyên nhân:} Nonce congestion và tài khoản chưa được fund đúng cách
\end{itemize}

\paragraph{Phân tích kết quả.}

Kết quả benchmark cho thấy hệ thống hoạt động ổn định ở mức 1--10 TPS với success rate 100\%. Ở mức 20 TPS, hệ thống gặp vấn đề do nonce congestion và việc quản lý tài khoản chưa tối ưu. Điều này phản ánh giới hạn thực tế của mạng blockchain khi xử lý tải cao, đặc biệt là khi sử dụng nhiều tài khoản đồng thời.

\subsubsection{Đánh giá hiệu năng}

\paragraph{Điểm mạnh.}

\begin{itemize}
    \item Hệ thống đạt 100\% success rate ở mức tải 1--10 TPS
    \item Latency ổn định ở khoảng 4 giây, phù hợp với yêu cầu giao dịch liên ngân hàng
    \item Không có lỗi giao dịch hoặc revert ở mức tải vừa phải
\end{itemize}

\paragraph{Điểm cần cải thiện.}

\begin{itemize}
    \item Cần tối ưu hóa quản lý nonce để hỗ trợ tải cao hơn (20+ TPS)
    \item Cần cải thiện cơ chế fund và quản lý tài khoản benchmark
    \item Có thể cần điều chỉnh block gas limit và block period để xử lý nhiều giao dịch hơn
\end{itemize}

\paragraph{Lý do lựa chọn.}

Việc thực hiện stress test với nhiều mức TPS khác nhau giúp xác định ngưỡng hiệu năng của hệ thống và các điểm nghẽn tiềm ẩn. Kết quả benchmark cung cấp dữ liệu thực nghiệm để đánh giá khả năng mở rộng của hệ thống và đưa ra các đề xuất cải thiện phù hợp với yêu cầu vận hành thực tế trong môi trường ngân hàng.



\subsection{Kết quả triển khai}



\paragraph{Tổng kết lựa chọn triển khai.}



Nhìn chung, các lựa chọn về môi trường, công nghệ và kiến trúc triển khai (Ubuntu + Docker + Besu, KSM tách biệt, TLS 1.3, Web GUI, PQC, PKI Registry, ZKP Prover OFF-CHAIN, và các contract riêng biệt) được thiết kế nhằm tái hiện sát bối cảnh một mạng blockchain liên ngân hàng có yêu cầu cao về bảo mật và riêng tư. Nhờ đó, hệ thống prototype không chỉ cho phép kiểm chứng các mục tiêu lý thuyết trong báo cáo tiến độ mà còn cung cấp cơ sở thực nghiệm để đánh giá khả năng áp dụng PQC và ZKP trong các kịch bản tài chính thực tế.







\end{document}

